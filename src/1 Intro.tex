\chapter{Introduction}

Information retrieval is the process of finding relevant material of unstructured nature from large collections. The ``material'' is usually documents, web pages, or multimedia content. Originally, information retrieval was something only a few professionals interacted with. Nowadays, hundreds of millions of people engage with information retrieval systems when they, for example, use a web search engine or search through their email.

The two key aspects considered when evaluating the quality of an IR system are \textbf{effectiveness} and \textbf{efficiency}. The first refers to the capability of the system to produce a satisfactory result; the second refers to how quickly it does so. To guarantee a certain level of quality, some operations are done offline, such as document indexing, feature processing (e.g., term frequency, metadata), training of a learn-to-rank model to produce the order in which documents will be shown to the user, and so on. Still, many operations must be done on-line, such as query expansion and processing, index and feature lookup, and usage of the ranking funcion. If we consider the example of a search engine (SE), we are used to getting back a response in a very short amount of time, despite the fact that in order to find the collection of documents presented to us, a lot of different operations must have been performed (i.e., the system must be efficient). Additionally, we also expect that those documents are the most relevant ones found in the collection, and that they are presented in the order of relevancy (the system must be effective). If those two ideas do not hold, we're unlikely to actually use the system for an extended amount of time.

The following chapters will go in detail about the different components of IR systems, and each will focus on how effectiveness and efficiency can be guaranteed. The key aspects that will be considered are:
\begin{itemize}
    \item \textbf{Language properties:} how does language influence retrieval? What does it mean to retrieve a piece of text? How are documents scored and presented?
    \item \textbf{Auxiliary data structures:} e.g., inverted indexes;
    \item \textbf{Query processing:} how are queries expanded from the form provided by the user into one which can be ``read'' by the system?
    \item \textbf{Data storage and compression:} how can data be compressed efficiently?
    \item \textbf{Learning-to-rank models:} how is machine learning used in an IR system to produce a ranking (based on available ranked data)?
    \item \textbf{Neural IR:} how can Deep Learning and specifically Large Language Models be used in IR?
\end{itemize}